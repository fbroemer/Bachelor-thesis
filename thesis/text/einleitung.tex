\newpage
\section{Einleitung}

Diese Arbeit hat das Ziel, eine Positionsbestimmung auf Basis von Schalllaufzeitmessungen zu entwickeln und aufzubauen. Sie soll die Grundlage sein für eine fehlerfreie Fahrt in Gebäuden mit dem hausinternen CE-Car, wobei die Positionsgenauigkeit dabei eine untergeordnete Rolle spielt \cite{src_CE_CAR}. Positionierungssysteme gibt es viele; allerdings sind diese nicht immer frei verfügbar, kostenintensiv und oft nur für den Outdoor-Bereich entwickelt worden \cite{src_INDOOR_OUTDOOR}. Ziel ist es, ein mobiles eingebettetes System für den Indoor-Bereich zu entwickeln -- mit dem Fokus auf geringe Kosten. Damit können Modellautos, Drohnen oder Roboter ausgestattet werden. Darüber hinaus muss das System ohne externe Dienste oder Netzanbindung funktionstüchtig sein. Zudem sollte es einfach erweiterbar sein, um der zukünftigen Entwicklung Schritt zu halten. Die Validierung erfolgt durch eine prototypische Anwendung.

\subsection{Aufbau der Arbeit}
Die Bachelorarbeit ist folgendermaßen gegliedert: Neben der Einleitung in Kapitel 1 werden in Kapitel 2 die theoretischen Grundlagen erläutert, sowie die verwendete Hardware und Software beschrieben. Darauf aufbauend beschreibt Kapitel 3 den Entwurf und danach in Kapitel 4 die Implementierung. Im Anschluss wird das System ausgewertet und evaluiert. Zum Schluss werden Probleme aufgezeigt, sowie ein Ausblick für mögliche Erweiterungen des Systems gegeben.