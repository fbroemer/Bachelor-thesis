\newpage
\section{Einleitung}

Diese Arbeit hat das Ziel, eine Positionsbestimmung auf Basis von Schalllaufzeitmessungen zu entwicklen und aufzubauen. Dies soll die Grundlage sein für eine fehlerfreie Fahrt in Gebäuden von dem Hausinternen CE-Car \cite{src_CE_CAR}. Die Positionsgenauigkeit spielt dabei eine untergeordnerte Rolle. Positionierungssysteme gibt es viele, allerdings sind diese nicht immer frei verfügbar, kostenintensiv und oft nur für den Outdoor Bereich entwickelt worden \cite{src_INDOOR_OUTDOOR}. Ziel ist es, ein mobiles eingebettetes System für den Indoorbereich zu entwicklen, mit dem Fokus auf geringe Kosten, sodass Modellautos, Drohnen oder Robotor damit ausgestattet werden können. Darüber hinaus muss das System ohne externe Dienste oder Netzanbindung funktionstüchtig sein. Zudem muss das System einfach erweiterbar sein, um der zukunftigen Entwicklung Stand zu halten. Die Validierung, erfolgt durch eine prototypische Anwendung.

\subsection{Aufbau der Arbeit}
Die Bachelorarbeit ist folgendermaßen gegliedert: Zuerst werden in Kapitel 2 die nötigen theoretischen Grundlagen erläutert, sowie die verwendete Hardware und Software beschrieben. Darauf aufbauend kommt Kapitel 3, indem die Umsetzung beschrieben wird. Im Anschlus wird das System evaluiert. Zum Schluss werden Probleme aufgezeigt, sowie ein Ausblick für mögliche Erweiterungen des Systems gegeben.
