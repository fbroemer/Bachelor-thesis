\newpage
\section{Probleme}
Dieser Abschnitt widmet sich den Problemen, die erst bei der Durchführung dieser Arbeit aufgetreten sind und vorher nicht abzuschätzen waren.


\subsection{Kommunikation Master -- Slave}
Die Idee der drahtlosen Kommunikation im Rahmen dieser Bachelorarbeit bestand darin, dass Steuerkommandos und Daten immer getrennt gesendet und immer auf 32-Bit aligned werden. Das hat zur Folge, dass Daten immer vom vorherigen Steuercode abhängig sind. Kommt der Steuercode nicht beim Empfänger an, können danach folgende Daten dem nicht zugeordnet bzw. korrekt interpretiert werden. Die Kommunikation war serialisert -- die nächsten Daten konnten nur interpretiert werden, wenn die vorherigen Daten korrekt ankamen. Da es häufig zu Verbindungsabbrüchen während der Kommunikation kam, wurde diese Kommunikationsvariante durch Structs abgelöst. Dabei enthält das Struct den Steuercode und die dazugehörigen Daten. Somit sind Steuercodes und Daten zusammen. Gehen nun Daten verloren, können folgende Daten weitherhin interpretiert werden, denn der dazugehörige Steuercode ist vorhanden/ bekannt. Die Kombination von Daten und Steuercodes ermöglicht eine variable Kommunikation. Man ist nicht mehr abhängig von den vorherigen Daten.

\subsection{Genauigkeit Zeitsynchronisation}
Bei der Zeitsynchronisation kam es immer wieder zu dem Problem das die Genauigkeit bei wiederholten mal, nicht besser wird. Somit ist es egal ob eine Zeitsynchronisation einmal oder mehrmals stattfindet. Dadurch das die Zeitsynchronisation nicht im \si{\mu}-Sekundenbereich auflöst, kann keine Messung ohne große Abweichung erfolgen. Die Abweichung kommt hauptsätlich durch die Aufrufe $getSystemTime()$ und $udp\_send\_packet()$ zustande. Denn die Systemzeit wird vor dem Aussenden vom Programm selber in das Datenpaket eingefügt und nicht von der Firmware des Funksenders. Weitherhin baut die Funktion $udp\_send\_packet()$ zuerst das UDP Paket zusammen, welches wiederrum Zeit kostet bis es gesendet wird. Eine weitere Vermutung ist, dass aufgrund von Störsignalen keine \si{\mu}-Sekundenauflösung erreicht werden kann für die Zeitsynchronisation. 

\subsection{Messgenauigkeit}
Theoretisch ist nach einer Zeitsynchronisation bekannt, um welche Zeit der Master dem Slave hinterherhängt, bzw. vorraus ist. Allerdings kam es bei den Distanzmessungen mit dem \board \platz dabei immer wieder zu großen Distanzschwankungen. Diese haben folgende Ursachen. Eine nicht im \si{\mu}-Sekundenbereich auflösende Zeitsynchronisation , Verzögerung der Messinstrumente und der Ablenkung des Schalls vom Aussenden bis zum Empfangen des Mikrofons.

