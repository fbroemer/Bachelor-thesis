\newpage
\section{Auswertung}

Nachdem die Implementierung abgeschlossen ist, widmen wir uns in diesem Kaptiel der Auswertung der Daten.

\subsection{Hardware}
Die aufgetretenen Abweichungen entstanden meistens, wenn die Peripherie Daten generiert hatte bzw. angesprochen wurde. Da viele dieser Abweichungen konstant sind, kann ein Offset für ein korrektes Ergebnis mit in das Ergebnis einfließen. Allerdings sind einige Abweichungen nicht erklärbar, was die Bestimmung eines Offset erschwert. Weiterhin zeigt sich, dass die verwendete Hardware nicht optimal auf dieses Problem abgestimmt ist. Dazu zählt der Tongeber und der \microphone . RIOT sieht sich zwar als Echtzeitbetriebssystem, allerdings sind zu viele Schichten zwischen dem laufenden Programm und der Hardware vorhanden, da ein Systemaufruf für die Abfrage der Systemzeit nicht über \SI{7,21}{\mu s} dauern. Dies verfälscht die Systemzeit, die eigentlich \si{\mu}-Sekunden genau sein soll. Des weiteren zeigt sich, je weniger Komponenten (Peripherie) verwendet werden, umso einfacher ist die Fehlersuche.

\subsection{Messergebnis}
Die Messergebnisse zeigen, dass eine Positionsbestimmung auf wie sich durchgeführt wurde nicht genau genug ist. Die Abweichung, die das Messergebnis hauptsächlich verfälscht, liegt beim Tongeber und dem Mikrofon. Weil kein Muster bei der Abweichung erkennbar ist, kann kein Offset bestimmt werden. Die Verwendung von fehlerbehafteten Bausteinen kann ausgeschlossen werden, da das Oszilloskop für alle verwendeteten Mikrofone die gleichen Abweichungen aufzeichnet.

\subsection{Software}

Die Software kann eine Positionsbestimmung durchführen; dabei ist sie allerdings auf korrekte Distanzmesswerte angewiesen. Da diese fehlen, kann die Software nur durch Unit-Tests validiert werden.
