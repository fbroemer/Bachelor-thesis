\newpage

\section{Ausblick}
Die Software kann in verschiedene Module eingeteilt werden. Damit ist es möglich, die Software zu erweitern, ohne die anderen Module zu verändern. Im folgenden werden Verbesserungsvorschläge für die einzelnen Module vorgestellt.

\paragraph{Kommunikation}\mbox{}\\
Für die Kommunikation zwischen Master und Slave könnte anstatt des UDP-Protokolls das TCP-Protokoll verwendet werden. Dadurch wird eine fehlerbehaftete Kommunikation verbessert, weil TCP mit ACKs arbeitet und zuerst ein Verbindungsaufbau erfolgt. Darüber hinaus könnte die Identifikation der Knoten im Netzwerk mit einer IP-Adresse erfolgen, anstatt auf Portnummern, wie es aktuell der Fall ist.

\paragraph{Zeitsynchronisation}\mbox{}\\
Damit eine Uhrensynchronisation im Nanosekundenbereich erreicht wird, muss die aktuelle Uhrzeit kurz vor dem Aussenden in das Datenpaket eingefügt werden. Dies sollte unabhängig vom System geschehen. Dafür muss die Systemzeit von der Firmware des Funkmoduls verwaltet werden. Wird die Systemzeit allerdings vom Betriebssystem verwaltet, gibt es immer eine Verzögerung zwischen dem Funktionsaufruf $getSystemTime()$ und dem Aussenden des Pakets. Um diese Verzögerungszeit zu eliminieren, sollte bei jedem Aussenden des Paketes die aktuelle Systemzeit angehängt werden.

\paragraph{Übertragungsmedium}\mbox{}\\
Für zukünftige Anwendungen könnte man neben dem Schall auch Radiosignale verwenden. Sie haben den Vorteil, dass das menschliche Gehör diese Frequenzen nicht wahrnimmt. Des weiteren breiten sich Radiosignale mit Lichtgeschwindigkeit aus.
\\
Darüber hinaus könnte eine Frequenzmodulationen durchgeführt werden, wodurch eine gleichzeitige Abfrage von mehreren Accesspoints möglich wäre. Dadurch ist eine schnellere Positionsbestimmung durchführbar. Radiosignale haben allerdings auch den Nachteil, dass eine Dämpfung z.B. bei Wänden stattfindet. Weitherhin muss bedacht werden, dass es zu Reflexionen, Streuung und Absorbation kommen kann \cite{src_RADIOSIGNALE}.

\paragraph{Messung}\mbox{}\\
Für die Positionsbestimmung müssen drei Gleichungen gelöst werden. Da MCUs nicht immer über eine FPU verfügen, kann eine MCU mit FPU verwendet werden. Als Alternative kann auch ein Raspberry-Pi genommen werden. Dieser ermöglicht die gemessenen Daten mit Octave oder Matlab weiter zu verarbeiten. Darüber hinaus ermöglicht Octave, Statistiken oder graphische Aufbereitung von Daten.

\paragraph{Hardware}\mbox{}\\
Aktuell ist der Lautsprecher und das Mikrofon über ein Steckbrett mit dem \board \platz verbunden. Es könnte auch eine Leiterplatte angefertigt werden, die auf das \board \platz aufgesteckt wird. Dies würde eventuelle Fehler beim Steckbrett vermeiden.  

\paragraph{Signalanalyse}\mbox{}\\
Um genauer herrauszufinden, wann das Signal beim Mikrofon ankommt, könnte eine Signalanalsye durchgeführt werden. Dies hätte den Vorteil, dass vortlaufend die Zeit gestoppt wird, sofort nachdem die erste Halbwelle der Sinusschwingung erkannt wird bzw. eine Periode vorbei ist. Dadurch wäre es dann möglich, einen definierten Zeitpunkt zu bestimmen, wann das Lautsprechersignal das Rauschen am \si{AUDIO}-Ausgang überlagert. Eine Vorraussetzung dafür wäre allerdings, dass ein Lautsprecher ein Signal dafür zurückgeben müsste, sofort nachdem die erste Halbwelle erreicht wurde.

\paragraph{Betriebssystem}\mbox{}\\
Anstatt das Betriebssystem RIOT zu verwenden, könnte auch ein eigenes Betriebssystem zur Anwendung zu kommen. Dies würde eine bessere Programmanalyse über das laufende Programm ermöglichen. Darüber hinaus könnten eventuelle Softwareroutinen, die nicht unbedingt essentiell sind, abgeschaltet bzw. gar nicht erst implementiert werden. Weiterhin käme auch in Betracht, das RIOT OS anzupassen bzw. zu erweitern.





