\newpage
\section{Zusammenfassung}

Das Ziel dieser Arbeit, eine Positionsbestimmung im Zentimeterbereich durchzuführen, konnte nicht vollständig erreicht werden. Ein Grund war die mir nur begrenzt zur Verfügung stehende Zeit sowie die Einschränkung bei der Leistungsfähigkeit der Hardware.
Die Arbeit hat verschiedene Aspekte der Positionsbestimmung untersucht: das Betriebssystem RIOT, die Hardware \microphone \platz und den \ultraschall. Dabei wurde die Verzögerung von Eingangs- und Ausgangssignalen untersucht.
\\
Die Ergebnisse haben gezeigt, dass RIOT sich aufgrund der starken Verzögerung des Funktionsaufrufs $getSystemTime()$ nicht für eine Positionsbestimmung eignet. Weiterhin wurde festgestellt, dass der verwendete Ultraschallsensor nicht die erforderlichen Anforderungen erfüllte, da er direkten Sichtkontakt benötigt und nur einen Sendekegel von \SI{15}{\degreeCelsius} hat. Eine Vermutung für die nicht vollständige Positionsbestimmung ist, dass der Schall zwischen dem Raum des Sound Detectors und dem Lautsprecher abgelenkt und nicht auf den direkten Weg zum Mikrofon transportiert wird. Für die Positionsbestimmung müssen Master und Slave eine synchrone Zeit haben. Um dies zu gewährleisten, wurde mit PTP eine Zeitsynchronisation implementiert. Dabei konnte nachgewiesen werden, dass ohne Hardwareunterstützung keine genaue Zeitsynchronisation im Nanosekundenbereich möglich ist, die für eine Positionsbestimmung notwendig ist. Dies spiegelt auch mein Ergebnis wieder. Um eventuelle neue Abweichungen diagnostizieren zu können, wäre es hilfreich, die beschriebenen Module für zukünftige anwendungsorientierte Forschungen zu wiederholen. Die Software liegt der Arbeit bei, sowie auf Github \cite{src_GITHUB_CODE_BA}\cite{src_genauigkeit_zeit_sync} .

