\newpage
\section{Anhang}

\appendix
\section{Mathematische Herleitung}
\label{sec:abcdef}
Diese Arbeit zeigt die Herleitung, nur für einen Punkt aus der Abbildung \ref{img:schwankungen}, denn die Herleitungen der anderen beiden Punkte unterscheiden sich nur in den Variabeln $x_{A}$, $y_{A}$ und $r_{A}$.
Zuerst wird eine Geradengleichung bestimmt, die durch die beiden Schnittpunkte von Kreis $A$ und $B$ gehen.

%\begin{equation}\label{eq:gerade_bestimmen}
%\begin{split}
%\RM{1} \quad (x-x_{A})^{2}+(y-y_{A})^{2} &= r_{A}^{2} \\
%\RM{2} \quad (x-x_{B})^{2}+(y-y_{B})^{2} &= r_{B}^{2} \\
%\cline{1-2}
%\RM{1} - \RM{2} \quad
%x \cdot (2 \cdot x_{B}-2 \cdot x_{A}) + y \cdot (2 \cdot y_{B}-2 \cdot y_{A}) + x_{A}^{2} - x_{B}^{2} + y_{A}^{2} - y_{B}^{2} &= r_{A}^{2} - r_{B}^{2} \\
%\RM{1} - \RM{2} \quad 
%x \cdot (2 \cdot x_{B} - 2 \cdot x_{A}) + y \cdot (2 \cdot y_{B} - 2 \cdot y_{A}) &= r_{A}^{2} - r_{B}^{2} - x_{A}^{2} + x_{B}^{2} - y_{A}^{2} + y_{B}^{2} \\
%\RM{1} - \RM{2} \quad y = \frac{ x \cdot (2 \cdot x_{A} - 2 \cdot x_{B}) +  r_{A}^{2} - r_{B}^{2} - x_{A}^{2} + x_{B}^{2} - y_{A}^{2} + y_{B}^{2} }{2 \cdot (y_{B} - y_{A})}
%\end{split}
%\end{equation}

\noindent
$
\RM{1} \hspace{5.7cm} (x-x_{A})^{2}+(y-y_{A})^{2} = r_{A}^{2} \\
\RM{2} \hspace{5.5cm} (x-x_{B})^{2}+(y-y_{B})^{2} = r_{B}^{2} \\
\RM{1}-\RM{2} \hspace{0.3cm} x \cdot (2 \cdot x_{B}-2 \cdot x_{A}) + y \cdot (2 \cdot y_{B}-2 \cdot y_{A}) + x_{A}^{2} - x_{B}^{2} + y_{A}^{2} - y_{B}^{2} = r_{A}^{2} - r_{B}^{2} 
$
\begin{equation}\label{eq:geradengleichung}
\hspace{1cm} y = \frac{ x \cdot (2 \cdot x_{A} - 2 \cdot x_{B}) +  r_{A}^{2} - r_{B}^{2} - x_{A}^{2} + x_{B}^{2} - y_{A}^{2} + y_{B}^{2} }{2 \cdot (y_{B} - y_{A})}
\end{equation}

Um nun die X-Koordinaten zu bekommen, setzen wir die Gleichung \ref{eq:geradengleichung} in Gleichung $\RM{1}$ ein und lösen nach $x$ auf.

\begin{equation}\label{eq:x_aufloesen}
\begin{split}
P = r_{A}^{2} - r_{B}^{2} - x_{A}^{2} + x_{B}^{2} - y_{A}^{2} + y_{B}^{2}
\\
(x-x_{A})^{2}+(y-y_{A})^{2}&=r_{A}^{2}\\
(x-x_{A})^{2} + (\frac{ x \cdot (2 \cdot x_{A} - 2 \cdot x_{B}) + P}{2 \cdot (y_{B} - y_{A})} - y_{A})^{2}&=r_{A}^{2}
\\
x^{2}(1+(\frac{x_{B}-x_{A}}{y_{B}-y_{A}})^{2}) + x (\frac{-P \cdot (x_{B} - x_{A})}{(y_{B} - y_{A})^{2}} + \frac{2 \cdot y_{A} \cdot (x_{B}-x_{A})}{y_{B} - y_{A}} - 2 \cdot x_{A}) \\
= r_{A}^{2} - y_{A}^{2} + \frac{y_{A} \cdot P}{y_{B}-y_{A}} - \frac{P^{2}}{4 \cdot (y_{B}-y_{A})^{2}} - x_{A}^{2}
\\
(1+(\frac{x_{B}-x_{A}}{y_{B}-y_{A}})^{2}) &= R
\\
(\frac{-P \cdot (x_{B} - x_{A})}{(y_{B} - y_{A})^{2}} + \frac{2 \cdot y_{A} \cdot (x_{B}-x_{A})}{y_{B} - y_{A}} &= S
\\
r_{A}^{2} - y_{A}^{2} + \frac{y_{A} \cdot P}{y_{B}-y_{A}} - \frac{P^{2}}{4 \cdot (y_{B}-y_{A})^{2}} - x_{A}^{2} &= T
\\
\\
x^{2} + x \cdot \frac{S - 2 \cdot x_{A}}{R} - \frac{T}{R} = 0
\end{split}
\end{equation}
Zum lösen einer quadratischen Gleichung wird die pq-Formel verwendet.
\begin{equation}\label{eq:x_aufloesen_pq_formel}
\begin{split}
p &= \frac{S - 2 \cdot x_{A}}{R}
\\
q &= \frac{T}{R}
\\
\\
x_{1,2} &= -\frac{p}{2} \pm \sqrt{(\frac{p}{2})^{2} -q}
\\
x_{1} &= -\frac{p}{2} + \sqrt{(\frac{p}{2})^{2} -q}
\\
x_{2} &= -\frac{p}{2} - \sqrt{(\frac{p}{2})^{2} -q}
\end{split}
\end{equation}

Nachdem die X-Koordinaten ermittelt worden sind, muss noch die Y-Koordinate für $x_{1}$ und $x_{2}$ bestimmt werden. Dafür werden die X-Koordinaten in die Geradengleichung \ref{eq:geradengleichung} eingesetzt.
\begin{equation}\label{eq:y_errechnen}
\begin{split}
\RM{1} \hspace{5.7cm} (x-x_{A})^{2}+(y-y_{A})^{2} = r_{A}^{2} \\
\hspace{1cm} y_{1} = \frac{ x_{1} \cdot (2 \cdot x_{A} - 2 \cdot x_{B}) +  r_{A}^{2} - r_{B}^{2} - x_{A}^{2} + x_{B}^{2} - y_{A}^{2} + y_{B}^{2} }{2 \cdot (y_{B} - y_{A})}
\\
\hspace{1cm} y_{2} = \frac{ x_{2} \cdot (2 \cdot x_{A} - 2 \cdot x_{B}) +  r_{A}^{2} - r_{B}^{2} - x_{A}^{2} + x_{B}^{2} - y_{A}^{2} + y_{B}^{2} }{2 \cdot (y_{B} - y_{A})}
\end{split}
\end{equation}

Die Schnittpunkte von Kreis $A$ und $B$ sind:
\begin{equation}\label{eq:schnittpunkte}
\begin{split}
(x_{1} | y_{1})
\\
(x_{2} | y_{2})
\end{split}
\end{equation}

Da nur ein Schnittpunkt auch in Kreis $C$ liegt, muss dieser bestimmt werden, sonst erhalten wir nicht minimalste Fläche die drei Kreise erzeugen. Um zu prüfen welcher Schnittpunkt in Kreis $C$ liegt, wird der Abstand vom Mittelpunkt des Kreises $C$ zu den Schnittpunkten aus Gleichung \ref{eq:schnittpunkte} berechnet.


\begin{equation}\label{eq:schnittpunkt_der_im_kreis_liegt}
\begin{split}
d_{1} = \sqrt{(x_{C} - x_{1})^2 + (y_{C} - y_{1})^{2}}
\\
d_{2} = \sqrt{(x_{C} - x_{2})^2 + (y_{C} - y_{2})^{2}}
\end{split}
\end{equation}

Wenn $d_{1} < d_{2}$ ist, dann ist der gesuchte Punkt $(x_{1} | y_{1})$. Falls die Ungleichung $d_{1} > d_{2}$ wahr ist, dann heißt der gesuchte Punkt $(x_{2} | y_{2})$.
\\
Diese Mathematische Vorgehen wird für Kreis $A-C$, und $B-C$ wiederholt.

\newpage
\section{Steuercodes}
Die folgende Tabelle listet alle vorhandenen Steuercodes auf.


\begin{table}[H]
\label{table:steuercodes}
\caption{Alle Steuercodes}
\centering
\begin{tabular}{ccc}
\hline
\multicolumn{1}{|c|}{Steurcode}          & \multicolumn{1}{c|}{\textit{unsigned int} Wert} & \multicolumn{1}{c|}{Beschreibung}                    \\ \hline
                                         &                                        &                                                      \\ \hline
\multicolumn{1}{|c|}{CODE\_MESSUNG}      & \multicolumn{1}{c|}{65}                & \multicolumn{1}{c|}{Messung durchführen}             \\ \hline
\multicolumn{1}{|c|}{CODE\_NOP}          & \multicolumn{1}{c|}{66}                & \multicolumn{1}{c|}{ACK anfordern}                      \\ \hline
\multicolumn{1}{|c|}{CODE\_ZEIT\_SYNC}   & \multicolumn{1}{c|}{67}                & \multicolumn{1}{c|}{SYNC-MSG senden} \\ \hline
\multicolumn{1}{|c|}{CODE\_ZEIT\_FOLLOW\_UP} & \multicolumn{1}{c|}{68}                & \multicolumn{1}{c|}{FOLLOW\_UP-MSG senden}         \\ \hline
\multicolumn{1}{|c|}{CODE\_ZEIT\_DELAY\_REQ} & \multicolumn{1}{c|}{69}                & \multicolumn{1}{c|}{DELAY\_REQ-MSG senden}         \\ \hline
\multicolumn{1}{|c|}{CODE\_ZEIT\_DELAY\_RESP} & \multicolumn{1}{c|}{70}                & \multicolumn{1}{c|}{DELAY\_RESP-MSG senden}   \\ \hline
\multicolumn{1}{|c|}{CODE\_READ\_T1\_T2\_T4} & \multicolumn{1}{c|}{71}                & \multicolumn{1}{c|}{Zurücksenden der Werte $t_{1}$, $t_{2}$, $t_{4}$}   \\ \hline
\multicolumn{1}{|c|}{CODE\_SERVER\_RESPONSE}         & \multicolumn{1}{c|}{72}                & \multicolumn{1}{c|}{ACK}                         \\ \hline

\end{tabular}
\end{table}
